
\subsection{Szótár}

\noindent\begin{xltabular}{\textwidth}{|l|X|}
	\hline
	\textbf{Kifejezés} & \textbf{Definíció} \\
	\hline
	\hline
	ágens & A vakcinák és vírusok közös neve. \\
	\hline
	ágens elkészítése & Minden egyes ágens esetén adott számú nukleotidból és adott számú aminosavból állítható elő egy ágnes, ha a virológus úgy dönt, hogy az akcióját arra használja fel, hogy előállítson egyet. Bármely mezőn képes erre. Csak olyat tud előállítani, aminek a genetikai kódját ismeri.  \\
	\hline
	ágens hatásos & Addig hatásos egy ágens, ameddig a hatóideje tart. \\
	\hline
	ágens hatóideje = lejárati ideje & Minden ágens esetén egy adott érték, minden körben csökken. \\
	\hline
	ágens visszadobása & Ha egy virológus rendelkezik kesztyűvel, akkor a körben első rá felhasznált ágens hatását automatikusan érvénytelennek tekinti, és az ágens hatása a küldőre fejtődik ki. Ezután a kesztyű elhasználódik. \\
	\hline
	ágnes elbomlik & Amikor lejár a hatóideje és már nem érvényesül többé a hatása.  \\
	\hline
	ágnes felhasználása & Egy ágens felhasználható magára a virológusra és bármely más virológusra is, ha egy mezőn állnak és még nem bomlott el. Ekkor a vírushoz tartozó hatás a vírus hatóidejének végéig érvényes lesz a virológusra, akire felhasználták. \\
	\hline
	akció & Minden körben dönthet a virológus, hogy mire használja fel az egy akcióját (ágnes elkészítése, védőfelszerelés felvétele, genetikai kód letapogatása, lopás, ágnes felhasználása, anyag begyűjtése). \\
	\hline
	aminosav & Egyfajta anyag, amiből előállítható ágens. \\
	\hline
	anyaggyűjtő képesség növelése & Alapvetően a zsák hatása okozza ezt: avirológus több alapanyagot tud egyszerre magánál tartani, mint korábban. \\
	\hline
	anyag & alapanyag \\
	\hline
	anyagok begyűjtése & Raktárakban lehetséges anyagot gyűjteni, a virológus felvesz annyi anyagot, amennyit csak tud a raktárból. Ez nem lehet több, mint amennyi eleve tárolva van a raktárban és nem lehet több, mint a virológus anyagtárolási korlátja. \\
	\hline
	anyagtárolási korlát & Egy virológus mennyi anyagot képes egyszerre magánál tárolni. \\
	\hline
	elfelejti a genetikai kódokat & A korábban már letapogatott genetikai kódokat elveszíti a virológus, többé nem tudja hogyan kell belőlük ágenseket előállítani. \\
	\hline
	eljutás & Amikor a virológus arra a mezőre lép, amelyik a célja.  \\
	\hline
	felejtő-vírus & A vírusokat 82.3\%-os hatásfokkal távol tartja: ha bárki (akár önmaga)  ágenst használ fel egy virológusra, akkor 82.3\% az esélye, hogy hatása érvénytelen lesz. \\
	\hline
	genetikai kód & A genetikai kódokat a laboratórium faláról lehet letapogatni a Genetikai kód letapogatása akció keretében. Ha egy játékos a virológusával a játékban szereplő összes genetikai kódot letapogatta, akkor megnyeri a játékot. A genetikai kódok segítségével lehet különböző ágenseket létrehozni.  \\
	\hline
	játékos & Maga a felhasználó, aki irányít egy virológust. \\
	\hline
	játéktér & A teljes pálya, vagyis a mezők összessége, amikre a virológusok léphetnek. \\
	\hline
	kesztyű & Egy védőfelszerelés, aminek segítségével egy ágens visszadobható annak felhasználójára. \\
	\hline
	kóborlás & A pályán való lépkedés, egy körben egy mezőt lehet lépni a virológus által választott mezőre. \\
	\hline
	kör & Áll egy lépésből és egy tetszőleges akcióból, ezután az ágensek lejárati ideje minden körben csökken. \\
	\hline
	kontrollálatlan, véletlenszerű mozgás & Annyi körön keresztül míg az okozó ágens le nem bomlik, a virológus nem tudja meghatározni merre lép, véletlenszerűen fog lépni egy mezőt. \\
	\hline
	laboratórium & Egyfajta mező, ahol található letapogatható genetikai kód. \\
	\hline
	lebénul & A virológus azon állapota, amikor nem tud lépni, sem akciót végezni. \\
	\hline
	genetikai kód letapogatása & Ha a virológus egy laboratóriumban van, akkor dönthet úgy, hogy megtanulja a laboratóriumban található ágens genetikai kódját. \\
	\hline
	megtanulni az összes felvehető genetikai kódot & Amikor minden játéktéren megtalálható genetikai kódot megtanult a virológus. \\
	\hline
	mező & A játéktér egysége, a virológusok mezőről-mezőre lépnek minden körben. Több fajtája van: szabad terület, laboratórium, raktár és óvóhely. \\
	\hline
	nukleotid & Egyfajta anyag, amiből előállítható ágens. \\
	\hline
	óvóhely & Egyfajta mező, ahol található ahonnan védőfelszerelés vehető fel a Védőfelszerelés felvétele akció keretében. \\
	\hline
	paralízis-vírus & Akire ilyen vírust használnak fel, lebénul. \\
	\hline
	raktár & Egyfajta mező, ahol anyagok találhatóak és begyűjthetőek az Anyag begyűjtése akció keretében. \\
	\hline
	vakcina & Jótékony hatású ágens, amely megvédi a hatása alatt álló személyt más ágensek hatásától. \\
	\hline
	város & A játéktér történet szerinti elnevezése. \\
	\hline
	véd a vírusok ellen & A vírus hatása nem érvényesül, ameddig a virológus védve van, az összes vírus lekerül róla, csak az lesz rá hatással, amit a vakcina lebomlása után használnak fel rá. \\
	\hline
	védőfelszerelés & felszerelés: Óvóhelyeken található, virológus által viselhető tárgy. Fajtái: zsák, kesztyű, védőköpeny. \\
	\hline
	védőfelszerelést viselni & Ha egy óvóhelyen tartózkodik a virológus, dönthet úgy, hogy az akciója keretében felveszi az ott található felszerelést. Ettől kezdve a hatása addig érvényesül rá, ameddig viseli. \\
	\hline
	védőköpeny & Egyfajta védőfelszerelés, ami az ágenseket 82.3\%-os hatásfokkal tartja távol. \\
	\hline
	virológus & karakter: A játék szereplői, akik a játéktér mezőin a körök során mozognak és akciókat használnak fel. \\
	\hline
	virológus elveszi a másik anyagkészletét és felszerelését & Amikor két virológus találkozik és az egyik le van bénulva, a másik elveheti a lebénult virológus anyagkészletét a Lopás akció keretében, de maximum a saját kapacitásáig, a többit otthagyja. Ezentúl elvehet tőle egy véletlenszerű védőfelszerelést is.  \\
	\hline
	virológusok találkoznak & A virológusok egy mezőre lépnek, érzékelik egymás jelenlétét. \\
	\hline
	vírus & Kártékony hatású ágens. Fajtái: vítustánc-vírus, paralízis-vírus, felejtő-vírus \\
	\hline
	vítustánc-vírus & Akire ilyen vírust használnak fel, kontrollálatlanul, véletlenszerűen mozog, ameddig a vírus el nem bomlik. \\
	\hline
	zsák & Egy védőfelszerelés, aminek segítségével megnő egy virológus anyaggyűjtő képessége. \\
	\hline
\end{xltabular}
