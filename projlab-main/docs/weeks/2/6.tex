
\subsection{Projekt terv}

\begin{itemize}
	\item[] \textbf{Határidők:} A féléves projekt feladatai és határidői lentebb találhatóak.
	\item[] \textbf{Felelősök:} Az egyes részfeladatok határidőre való leadása minden csapattag közös felelőssége. A részfeladatokon belüli feladatmegosztást az egyes feladatokra nézve határozzuk meg, ennek meghatározása általában a heti értekezleten történik meg (lásd 2.7 Napló).
	\item[] \textbf{Támogató eszközök:} A projektmenedzsmentre a Click Up webes felületét használjuk. A dokumentumot a közös szerkeszthetőség érdekében először Google Docs alkalmazással írjuk, utána a formázáshoz \LaTeX-et használunk. A programkód írására az IntelliJ IDEA (esetleg Visual Studio Code) fejlesztőkörnyezet lehetőleg legfrissebb változatát használjuk. Kapcsolattartásra a Telegram-ot, az értekezletek lebonyolítására a Discord alkalmazást használjuk.
	\item[] \textbf{Verziókövetés:} Verziókövetésre git-et és a Schönherzes GitLab-ot tervezzük használni.
\end{itemize}


\subsubsection{Ütemezés}
\noindent\begin{tabularx}{\textwidth}{|l|l|X|}
	\hline
	Hét                                             & Határidő                                                                   & Feladat                              \\ \hline
	\hline
	2                                               & febr. 28.                                                                  & Követelmény, projekt, funkcionalitás \\ \hline
	3                                               & márc. 7.                                                                   & Analízis modell (I. változat)        \\ \hline
	4                                               & \begin{tabular}[c]{@{}l@{}}márc. 16.\\ (szerdai laboralkalom)\end{tabular} & Analízis modell (II. változat)       \\ \hline
	5                                               & márc. 21.                                                                  & Szkeleton tervezése                  \\ \hline
	6                                               & márc. 28.                                                                  & Szkeleton elkészítése                \\ \hline
	7                                               & ápr. 4.                                                                    & Prototípus koncepciója               \\ \hline
	8                                               & ápr. 11.                                                                   & Részletes tervek                     \\ \hline
	\begin{tabular}[c]{@{}l@{}}9\\ 10\end{tabular}  & ápr. 25.                                                                   & Prototípus elkészítése               \\ \hline
	11                                              & máj. 2.                                                                    & Grafikus változat tervei             \\ \hline
	\begin{tabular}[c]{@{}l@{}}12\\ 13\end{tabular} & máj. 16.                                                                   & Grafikus változat készítése          \\ \hline
	14                                              & máj. 18.                                                                   & Egyesített dokumentáció              \\ \hline
\end{tabularx}
