
\newcounter{magicsorszam}

\newcommand\azonosito{\stepcounter{magicsorszam}R\padzeroes[2]{\decimal{magicsorszam}}}


\subsection{Követelmények}

\subsubsection{Funkcionális követelmények}

\noindent\begin{xltabular}{\textwidth}{| X | p{4cm} | p{2cm} | p{1.8cm} | p{1.3cm} | p{2cm} | X |}
	\hline
	\textbf{Azonosító} &
	\textbf{Leírás} &
	\textbf{Ellenőrzés} &
	\textbf{Prioritás} &
	\textbf{Forrás} &
	\textbf{Use-case} &
	\textbf{Komment} \\
	\hline
	\hline

	\azonosito &
	A játékot humán játékosok játszhatják. &
	bemutatás &
	fontos &
	csapat &
	View interface, Move character, Scan genetic code, Collect material, Create agent, Use agent, Pick up equipment, Steal equipment &
	- \\
	\hline

	\azonosito &
	A játékot egyszerre többen játszhatják. &
	bemutatás &
	fontos &
	csapat &
	View interface, Move character, Scan genetic code, Collect material, Create agent, Use agent, Pick up equipment, Steal equipment &
	- \\
	\hline

	\azonosito &
	Akár egy résztvevő is indíthat játékot. &
	bemutatás és kiértékelés &
	opcionális &
	csapat &
	View interface &
	- \\
	\hline

	\azonosito &
	Minden játékoshoz tartozik egy karakter, akit irányíthat. &
	bemutatás &
	fontos &
	csapat &
	View interface, Move character, Scan genetic code, Collect material, Create agent, Use agent, Pick up equipment, Steal equipment &
	- \\
	\hline

	\azonosito &
	A játék körökre van osztva. &
	bemutatás &
	fontos &
	csapat &
	Control time &
	- \\
	\hline

	\azonosito &
	A játékosok egymás után kerülnek sorra egy a játék elején meghatározott sorrendben. &
	bemutatás &
	fontos &
	csapat &
	Control time &
	- \\
	\hline

	\azonosito &
	Egy körben egy játékos két dolgot tehet: mozgás és akció. &
	bemutatás &
	fontos &
	csapat &
	View interface, Move character, Scan genetic code, Collect material, Create agent, Use agent, Pick up equipment, Steal equipment, Control time &
	- \\
	\hline

	\azonosito &
	Mozgás: választhat a jelenlegi mezőjével szomszédosak közül, hogy melyikre szeretne lépni. &
	bemutatás &
	alapvető &
	csapat &
	View interface, Move character &
	- \\
	\hline

	\azonosito &
	Mozogni nem kötelező. &
	bemutatás &
	opcionális &
	csapat &
	View interface, Move character &
	- \\
	\hline

	\azonosito &
	Mozgás után lehetősége van legfeljebb egy akciót végrehajtani az alábbiak közül: genetikai kód letapogatása, anyag begyűjtése, védőfelszerelés felvétele, ágens elkészítése,  ágens felhasználása, lopás. &
	bemutatás &
	fontos &
	csapat &
	View interface, Scan genetic code, Collect material, Create agent, Use agent, Pick up equipment, Steal equipment &
	- \\
	\hline

	\azonosito &
	Az anyagnak két típusát különböztetjük meg: aminosav és nukleotid. &
	bemutatás &
	alapvető &
	feladatkiírás &
	Collect material &
	- \\
	\hline

	\azonosito &
	Anyagok raktár mezőkön gyűjthetők össze. &
	bemutatás &
	alapvető &
	feladatkiírás &
	Collect material &
	- \\
	\hline

	\azonosito &
	Az Anyag begyűjtése akciót választva lehet anyagot felvenni, amennyiben az adott raktár területén található anyag. &
	bemutatás &
	fontos &
	csapat &
	View interface,Collect material &
	- \\
	\hline

	\azonosito &
	Egy virológusnál egyszerre csak bizonyos mennyiségű anyag lehet. &
	bemutatás és kiértékelés &
	alapvető &
	feladatkiírás &
	Collect material &
	- \\
	\hline

	\azonosito &
	Ha a birtokolt anyag mennyiség eléri a korlátot, a virológus nem gyűjthet be több anyagot &
	bemutatás és kiértékelés &
	alapvető &
	feladatkiírás &
	Collect material &
	- \\
	\hline

	\azonosito &
	Az ágensek is kétfélék lehetnek: vírusok vagy vakcinák. &
	bemutatás &
	alapvető &
	feladatkiírás &
	Create agent, Use agent &
	- \\
	\hline

	\azonosito &
	A genetikai kódokat a virológusok a laboratóriumok falán tudják letapogatni a Genetikai kód letapogatása akció keretében. &
	bemutatás &
	alapvető &
	feladatkiírás &
	Scan genetic code &
	- \\
	\hline

	\azonosito &
	Egy kódot akárhány virológus letapogathat. &
	bemutatás és kiértékelés &
	fontos &
	csapat &
	Scan genetic code &
	- \\
	\hline

	\azonosito &
	Ha egy ágensnek megismeri valaki a genetikai kódját, majd összegyűjti a hozzá szükséges anyagokat, az Ágens elkészítése akciót választva elkészítheti. &
	bemutatás &
	alapvető &
	feladatkiírás &
	View interface, Scan genetic code, Create agent &
	- \\
	\hline

	\azonosito &
	Az Ágens felhasználása akcióval a virológus rákenheti egy másik virológusra vagy akár saját magára. &
	bemutatás &
	alapvető &
	feladatkiírás &
	Use agent &
	- \\
	\hline

	\azonosito &
	Az ágensek hatása előre meghatározott körig tart, utána érvényüket vesztik. &
	bemutatás &
	fontos &
	csapat &
	Use agent, Control time &
	- \\
	\hline

	\azonosito &
	Vítustánc-vírus: amíg a vírus hatása alatt áll egy játékos, a karakterét nem lehet irányítani, a szomszédos mezők közül egyet véletlenszerűen választva lép. &
	bemutatás &
	alapvető &
	feladatkiírás &
	Use agent, Move character, Control time &
	- \\
	\hline

	\azonosito &
	Amikor valakire alkalmazzák a vakcinát, azon megszűnnek hatni a már rákent és az ezután alkalmazott vírusok is adott számú körig. &
	bemutatás és kiértékelés &
	alapvető &
	feladatkiírás &
	Use agent, Control time &
	- \\
	\hline

	\azonosito &
	A játékos, akin a paralízis-vírust alkalmazzák, adott számú körig egy helyben marad, nem tud se lépni, se akciót végrehajtani. &
	bemutatás &
	alapvető &
	feladatkiírás &
	Use agent, Move character, Control time &
	- \\
	\hline

	\azonosito &
	Akire a felejtő-vírust alkalmazzák, elfelejti az összes genetikai kódot, amit addig megtanult. &
	bemutatás és kiértékelés &
	alapvető &
	feladatkiírás &
	Use agent &
	- \\
	\hline

	\azonosito &
	A játékosok védőfelszereléseket is gyűjthetnek a játékban. &
	bemutatás &
	alapvető &
	feladatkiírás &
	Pick up equipment &
	- \\
	\hline

	\azonosito &
	Védőfelszerelések az óvóhely mezőkről szedhetők fel a Védőfelszerelés felvétele akció választásával. &
	bemutatás &
	alapvető &
	feladatkiírás &
	View interface, Pick up equipment &
	- \\
	\hline

	\azonosito &
	Egy virológusnál egyszerre maximum három védőfelszerelés lehet. &
	bemutatás és kiértékelés &
	alapvető &
	feladatkiírás &
	Pick up equipment &
	- \\
	\hline

	\azonosito &
	A védőfelszerelések ellophatók egymástól a Lopás akció keretében. &
	bemutatás &
	alapvető &
	feladatkiírás &
	Steal equipment &
	- \\
	\hline

	\azonosito &
	A Lopás akció akkor lehet sikeres, ha a megtámadott játékos bénult állapotban van. &
	bemutatás és kiértékelés &
	alapvető &
	feladatkiírás &
	Use agent, Steal equipment &
	- \\
	\hline

	\azonosito &
	A lopást indító virológushoz egy új védőfelszerelés és maximálisan elvehető mennyiségű anyag kerül a játékostársa készletéből. &
	bemutatás &
	fontos &
	csapat &
	Steal equipment &
	- \\
	\hline

	\azonosito &
	Az esetek 82,3\%-ban a védőköpenyt viselőre rákent ágensek hatástalanok lesznek. A saját magára kent ágensek esetén is azonos a hatása. &
	bemutatás &
	alapvető &
	feladatkiírás &
	Use agent &
	- \\
	\hline

	\azonosito &
	A zsák  növeli a virológus anyag-befogadóképességét, így több anyagot tud magánál tartani. &
	bemutatás és kiértékelés &
	alapvető &
	feladatkiírás &
	Collect material &
	- \\
	\hline

	\azonosito &
	A kesztyű visszaveri a felvétel után a virológusra dobott ágenst, annak hatásától függetlenül. &
	bemutatás &
	alapvető &
	feladatkiírás &
	Use agent &
	- \\
	\hline

	\azonosito &
	A kesztyű egy használat után elveszik. &
	bemutatás és kiértékelés &
	fontos &
	csapat &
	Use agent &
	- \\
	\hline

	\azonosito &
	Egy adott típusú védőfelszerelésből több példány is megtalálható a játékban. &
	bemutatás &
	fontos &
	csapat &
	Pick up equipment &
	- \\
	\hline

	\azonosito &
	Egy játékos minden védőfelszerelés típusból legfeljebb egyet birtokolhat. &
	bemutatás és kiértékelés &
	fontos &
	csapat &
	Pick up equipment &
	- \\
	\hline

	\azonosito &
	A Lopás és az Ágens felhasználása akcióra akkor van lehetőség, ha egy virológus más virológusokkal találkozik egy játékmezőn. &
	bemutatás &
	alapvető &
	feladatkiírás &
	Steal equipment, Use agent &
	- \\
	\hline

	\azonosito &
	Egy területtel legalább egy, de véges számú más terület lehet szomszédos. &
	bemutatás &
	alapvető &
	feladatkiírás &
	View interrface, Move character &
	- \\
	\hline

	\azonosito &
	A játékot az a játékos nyeri, aki az összes a játékban szereplő ágens genetika kódját először megtanulja. &
	bemutatás és kiértékelés &
	alapvető &
	feladatkiírás &
	Scan genetic code &
	- \\

	\hline
\end{xltabular}

\subsubsection{Erőforrásokkal kapcsolatos követelmények}

\noindent\begin{tabularx}{\textwidth}{| l | X | l | l | l | l | l |}
	\hline
	\textbf{Azonosító}                                                                              &
	\textbf{Leírás}                                                                                 &
	\textbf{Ellenőrzés}                                                                             &
	\textbf{Prioritás}                                                                              &
	\textbf{Forrás}                                                                                 &
	\textbf{Komment}                                                                                  \\

	\hline
	\hline

	\azonosito                                                                                      &
	A futtató számítógépen szükség van JVM-re, amely futtatásához rendelkezik megfelelő hardverrel. &
	bemutatás                                                                                       &
	alapvető                                                                                        &
	csapat                                                                                          &
	-                                                                                                 \\

	\hline
\end{tabularx}

\subsubsection{Átadással kapcsolatos követelmények}

\noindent\begin{tabularx}{\textwidth}{| l | X | l | l | l | l | l |}
	\hline
	\textbf{Azonosító}                                   &
	\textbf{Leírás}                                      &
	\textbf{Ellenőrzés}                                  &
	\textbf{Prioritás}                                   &
	\textbf{Forrás}                                      &
	\textbf{Komment}                                       \\

	\hline
	\hline

	\azonosito                                           &
	A programnak működnie kell az adott virtuális gépen. &
	bemutatás                                            &
	alapvető                                             &
	megrendelő                                           &
	-                                                      \\

	\hline
\end{tabularx}

\subsubsection{Egyéb nem funkcionális követelmények}

\noindent\begin{tabularx}{\textwidth}{| l | X | l | l | l | l | l |}
	\hline
	\textbf{Azonosító}                                                                        &
	\textbf{Leírás}                                                                           &
	\textbf{Ellenőrzés}                                                                       &
	\textbf{Prioritás}                                                                        &
	\textbf{Forrás}                                                                           &
	\textbf{Komment}                                                                            \\

	\hline
	\hline


	\azonosito                                                                                &
	A programnak működnie kell Windows operációs rendszereken.                                &
	bemutatás                                                                                 &
	alapvető                                                                                  &
	csapat                                                                                    &
	-                                                                                           \\
	\hline


	\azonosito                                                                                &
	A programnak működnie kell Linux operációs rendszereken.                                  &
	bemutatás                                                                                 &
	fontos                                                                                    &
	csapat                                                                                    &
	-                                                                                           \\
	\hline


	\azonosito                                                                                &
	A kiértékeléssel ellenőrizendő funkcionális követelményeknek tesztelhetőnek kell lenniük. &
	bemutatás                                                                                 &
	fontos                                                                                    &
	csapat                                                                                    &
	-                                                                                           \\
	\hline


	\azonosito                                                                                &
	A felhasználóknak tudni kell kezelni a számítógép perifériáit.                            &
	bemutatás                                                                                 &
	alapvető                                                                                  &
	csapat                                                                                    &
	-                                                                                           \\

	\hline
\end{tabularx}

