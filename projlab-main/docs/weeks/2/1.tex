\subsection{Bevezetés}

\subsubsection{Cél}

A dokumentum célja az Indonesian Monkeys Social Club csapat projektindító gyűlésén megbeszéltek rögzítése, a követelmények véglegesítése, a készítendő szoftver jellemzőinek pontosítása, az azzal kapcsolatos elvárásaink leírása.

\subsubsection{Szakterület}

A szoftver célja, hogy a “Szoftver projekt laboratórium” (BMEVIIIAB06) tárgy követelményeit teljesítse, a szoftvert tesztelő oktatók és hallgatók elvárásait kielégítse. A szoftver másodlagos célja, hogy kielégítse azok igényeit, akik a globális pandémia alatt egy hasonló tematikájú fantasy játékkal szeretnének játszani.

\subsubsection{Definíciók, rövidítések}

A továbbiakban a dokumentumban szereplő azon definíciók, rövidítések szerepelnek, amik további kifejtést igényelnek:

\begin{itemize}
	\item[] \textbf{pl.} például
	\item[] \textbf{IMSC} Indonesian Monkeys Social Club
\end{itemize}

\subsubsection{Hivatkozások}

A Szoftver Projekt Laboratórium tantárgy 2022-es évi feladatkiírása képezi a játék tematikájának és szabályrendszerének alapját:

https://www.iit.bme.hu/targyak/BMEVIIIAB02

A sablon kitöltéséhez a korábbi évek feladatkiírásai, és az ütemterv is segítségünkre volt:

https://www.iit.bme.hu/targyak/BMEVIIIAB02/ütemterv-határidők

A \LaTeX\hspace{0.5em}dokumentációk formázásához a korábbi években elkészített dokumentációkból vettünk ihletet:

https://git.sch.bme.hu/insert-epic-projlab-team-name-here/projlab/-/tree/master/docs

\subsubsection{Összefoglalás}

A továbbiakban részletesen ismertetésre kerülnek a játék funkciói, a játék szabályrendszere, valamint a szoftver megvalósításával kapcsolatos fontosabb tudnivalók is.
